\documentclass[runningheads]{llncs}

%\usepackage{geometry}
\usepackage{multirow}
\usepackage{amsmath}
\usepackage{amssymb}
%\usepackage{amsthm}
\usepackage{xcolor}
\usepackage{graphicx}
\usepackage[utf8]{inputenc}
\usepackage{siunitx}

\usepackage[numbers]{natbib}
\renewcommand\bibsection{\section*{\refname}\small\renewcommand\bibnumfmt[1]{##1.}}

\usepackage{colortbl}
\usepackage{array}
\newcommand{\PreserveBackslash}[1]{\let\temp=\\#1\let\\=\temp}
\newcolumntype{C}[1]{>{\PreserveBackslash\centering}p{#1}}
\newcolumntype{R}[1]{>{\PreserveBackslash\raggedleft}p{#1}}
\newcolumntype{L}[1]{>{\PreserveBackslash\raggedright}p{#1}}
\newcommand\Tstrut{\rule{0pt}{2.6ex}}         % = `top' strut
\newcommand\Bstrut{\rule[-0.9ex]{0pt}{0pt}}   % = `bottom' strut

\usepackage{hyperref}
\hypersetup{
  colorlinks   = true, %Colours links instead of ugly boxes
  urlcolor     = black, %Colour for external hyperlinks
  linkcolor    = blue, %Colour of internal links
  citecolor    = blue  %Colour of citations
}

%%%%%%%%%%%%%%%%%%%%%%%%%%%%%%%%%%%%%%%%%%%%%%%%%%%%%%%%%%%%%%%%%%%%%%%%%%%%%%%%

\newcommand{\escrawl}{\texttt{SpanishCrawl}}

\newcommand{\defn}[1]{\emph{#1}}

\newcommand{\fixme}[1]{\textbf{FIXME:} #1}
\newcommand{\ignore}[1]{}

%%%%%%%%%%%%%%%%%%%%%%%%%%%%%%%%%%%%%%%%%%%%%%%%%%%%%%%%%%%%%%%%%%%%%%%%%%%%%%%%

\begin{document}

\title{Translating Words between 25 Spanish Dialects}
%
%\titlerunning{Abbreviated paper title}
% If the paper title is too long for the running head, you can set
% an abbreviated paper title here
%
\author{Author blinded for peer review}
%
%\authorrunning{Blinded for peer review}
% First names are abbreviated in the running head.
% If there are more than two authors, 'et al.' is used.
%
\institute{Institute blinded for peer review}
%
\maketitle              % typeset the header of the contribution

\begin{abstract}
Word vectors have become one of the most important techniques in natural language processing since Mikolov et.\ al.\ introduced the skipgram model in 2013.
To train our model, we introduce the \escrawl\ dataset,
which contains 2.4 billion words labeled with 25 different Spanish dialects.

\keywords{word vectors \and tensor decompositions \and machine translation.}
\end{abstract}

%%%%%%%%%%%%%%%%%%%%%%%%%%%%%%%%%%%%%%%%

\section{Introduction}

Spanish is spoken in dozens of countries around the world
%More people speak Spanish as their native language than any other language except Chinese.

The largest previously existing dataset of Spanish language text is Spanish Billion Word Corpus \citep{cardellino2016billion}.
This dataset is not suitable for out purposes because the text is not labelled by dialect.
\defn{country-code top level domain} (ccTLD)

%%%%%%%%%%%%%%%%%%%%%%%%%%%%%%%%%%%%%%%%

\section{The \escrawl~Dataset}

The \escrawl\ corpus contains 2.4 billion natural language Spanish words gathered from 33 million web pages across 19 thousand domains.
Each sentence in the corpus is labeled with a country-of-origin according to the \defn{country-code top level domain} (ccTLD).
Each country is assigned a unique ccTLD by the Internet Assigned Numbers Authority (IANA),
and that country has the authority to determine which web pages can be hosted under its ccTLD.
We assume that the Spanish text hosted under a country's ccTLD is linguistically representative of the dialectical Spanish used in that country.
For example, the Cuban newspaper \emph{Granma} hosts their website at \url{www.granma.cu};
this newspaper writes for Cubans in the Cuban dialect and has the Cuban ccTLD \url{cu}.

\subsection{Comparison to Other Corpora}

\subsection{Construction Methodology}

%For example, the newspaper \emph{Granma} has the domain name \url{www.granma.cu}
%and the newspaper \emph{La Jornada} has the domain name \url{www.jornada.com.mx}.
%We can use the ccTLDs of these domains to correctly infer that the Spanish text in \emph{Granma} is of the Cuban dialect whereas the Spanish text in \emph{La Jornada} is written in the Mexican dialect.
%This method cannot find all Spanish language dialectical text online.
%For example, the Mexican newspaper \emph{Prensa Escrita} has domain name \url{www.prensaescrita.com},
%which is not associated with any ccTLD,
%and so our method will ignore this potential information source.

We identified 38 ccTLDs as candidates for crawling.
The list contains all countries with Spanish as their official language (plus Puerto Rico),
all countries in the mainland Americas and C

We use Google to generate seed URLs for the crawl that contain a mixture of newspaper, university, and government websites.
For each ccTLD, we perform the following three searches:
\begin{verbatim}
    site:.ccTLD noticias
    site:.ccTLD universidad
    site:.ccTLD gobierno
\end{verbatim}
where \texttt{ccTLD} is replaced by the actual ccTLD.
The \texttt{site:} operator ensures that Google only returns results from the appropriate ccTLD.
The seed URLs for that ccTLD are then the top 500 results for each search query.

We performed a crawl for each ccTLD separately on their own machines using Python's \texttt{scrapy} web crawler.
For each page in the crawl, we use the \texttt{langdetect} library to identify the page's language.
For web
Many of the domains we crawled have translations from Spanish into English or other languages,
and to
If the webpage is written in Spanish,
we add the text to our dataset and recursively crawl all referenced links.
Many Spanish webpages have versions translated into English and other languages,
and to avoid wasting time crawling these non-Spanish webpages,
we do not follow links for non-Spanish pages.
We let these crawlers run for 3 weeks.

we used the \texttt{Goose3} library to extract the article's text content from the HTML.
\texttt{Goose3} uses a sophisticated algorithm to ensure that irrelevant HTML content such as navigation bars do not get included in the returned text,
and thus the returned text is of high quality.

\begin{table}
\center
\begin{tabular}{ll R{0.7in}R{1in}R{1in}}
\textbf{ccTLD}~~ & \textbf{Country} & \textbf{Domains} & \textbf{URLs} & \textbf{Words} \\
\hline
\Tstrut\texttt{~ar} & Argentina &  $\SI{2346}{}$ & $\SI{1344007}{}$ & $\SI{98019327}{}$ \\
\texttt{~bo} & Bolivia &  $\SI{163}{}$ & $\SI{3186350}{}$ & $\SI{4899340}{}$ \\
\texttt{~cl} & Chile &  $\SI{2529}{}$ & $\SI{371222}{}$ & $\SI{57318702}{}$ \\
\texttt{~co} & Colombia &  $\SI{382}{}$ & $\SI{340951}{}$ & $\SI{51391569}{}$ \\
\texttt{~cr} & Costa Rica &  $\SI{396}{}$ & $\SI{195835}{}$ & $\SI{12624659}{}$ \\
\texttt{~cu} & Cuba &  $\SI{817}{}$ & $\SI{3108860}{}$ & $\SI{726121030}{}$ \\
\texttt{~do} & Dominican Republic &  $\SI{647}{}$ & $\SI{1202656}{}$ & $\SI{14263274}{}$ \\
\texttt{~ec} & Ecuador &  $\SI{618}{}$ & $\SI{265144}{}$ & $\SI{18287567}{}$ \\
\texttt{~es} & Spain &  $\SI{2416}{}$ & $\SI{975356}{}$ & $\SI{127258063}{}$ \\
\texttt{~gq} & Equatorial Guinea &  $\SI{514}{}$ & $\SI{287058}{}$ & $\SI{36165880}{}$ \\
\texttt{~gt} & Guatemala &  $\SI{70}{}$ & $\SI{127244}{}$ & $\SI{5241832}{}$ \\
\texttt{~hn} & Honduras &  $\SI{151}{}$ & $\SI{1852839}{}$ & $\SI{58781290}{}$ \\
\texttt{~mx} & Mexico &  $\SI{3427}{}$ & $\SI{865430}{}$ & $\SI{332296803}{}$ \\
\texttt{~ni} & Nicaragua &  $\SI{158}{}$ & $\SI{280751}{}$ & $\SI{107425790}{}$ \\
\texttt{~pa} & Panama &  $\SI{164}{}$ & $\SI{398516}{}$ & $\SI{16239530}{}$ \\
\texttt{~pe} & Peru &  $\SI{1357}{}$ & $\SI{971999}{}$ & $\SI{58263651}{}$ \\
\texttt{~pr} & Puerto Rico &  $\SI{49}{}$ & $\SI{3228105}{}$ & $\SI{4666549}{}$ \\
\texttt{~py} & Paraguay &  $\SI{762}{}$ & $\SI{594761}{}$ & $\SI{56558666}{}$ \\
\texttt{~sv} & El Salvador &  $\SI{285}{}$ & $\SI{559651}{}$ & $\SI{28437744}{}$ \\
\texttt{~uy} & Uruguay &  $\SI{1200}{}$ & $\SI{1759446}{}$ & $\SI{48906640}{}$ \\
\texttt{~ve} & Venezuela &  $\SI{622}{}$ & $\SI{4084897}{}$ & $\SI{248668439}{}$ \\
\arrayrulecolor{lightgray}\hline
\arrayrulecolor{black}\Tstrut\texttt{~br} & Brazil &  $\SI{27}{}$ & $\SI{302091}{}$ & $\SI{137977887}{}$ \\
\texttt{~bz} & Belize &  $\SI{7}{}$ & $\SI{217645}{}$ & $\SI{3883216}{}$ \\
\texttt{~pt} & Portugal &  $\SI{10}{}$ & $\SI{3826640}{}$ & $\SI{20166915}{}$ \\
\texttt{~us} & United States &  $\SI{9}{}$ & $\SI{2729260}{}$ & $\SI{130693418}{}$ \\
\hline
\Tstrut\textbf{Total} &&  $\SI{19126}{}$ & $\SI{33076714}{}$ & $\SI{2404557781}{}$
\end{tabular}

\vspace{0.1in}
\caption{Summary of the \escrawl~dataset.}
\label{table:escrawl}
\end{table}

%%%%%%%%%%%%%%%%%%%%%%%%%%%%%%%%%%%%%%%%

\ignore{
\citet{conneau2017word} propose word embeddings without parallel data,
and \citet{lample2017unsupervised} use these embeddings to perform fully unsupervised language translation.
We propose using the $\ell1$ \fixme{oblique?} procrustes problem \citep{trendafilov2003ell,trendafilov2004ell}.

\citet{salloum2011dialectal,salloum2013dialectal} introduce rules based systems for converting dialectical arabic into standard arabic.
They propose a method for measuring the quality of their results by measuring how well it improves the BLEU score of a subsequent translation into English.

\citet{mager2018challenges} surveys the little work that has been done to study indigenous Latin American languages.
Studying Spanish dialects may help us improve our understanding of these languages because of the strong and localized influences.

\citet{costa2018neural} studies translation between European Portuguese to Brazilian Portuguese.
They claim this is the first translation study between language dialects using deep learning,
and provide references to systems using older techniques.

\citet{xu2012paraphrasing} develop a method for translating from Shakespearean English into modern English,
but their method requires parallel text corpora.
Parallel corpora for Spanish dialects exist only at the most course level.
For example, Harry Potter has been translated into a Castilian dialect and Latin American dialect.

\citet{potthast2016author} survey author identification methods and methods for obfuscating author identity.
The dialect translation task can be seen as an obfuscation task.
Discusses the PAN2016 workshop on author identification/obfuscation.

\citet{zissman1996automatic} build a system for differentiating between Cuban and Peruvian written dialects.
\citet{yanguas1998incorporating;caballero2009multidialectal} study spoken dialectical variations.

VarDial Workshop

\citet{zbib2012machine} create parallel corpora for Egyptian-Arabic to English and Levantine-Arabic to English,
and study how dialectical differences in Arabic can affect Arabic to English translation.
\fixme{View citations on google scholar for more Arabic.}
}

%%%%%%%%%%%%%%%%%%%%%%%%%%%%%%%%%%%%%%%%%%%%%%%%%%%%%%%%%%%%%%%%%%%%%%%%%%%%%%%%
\small
\bibliographystyle{plain}
\bibliography{main}

\end{document}
